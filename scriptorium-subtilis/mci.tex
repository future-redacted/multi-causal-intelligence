
\documentclass[10pt]{article}
\usepackage{amsmath,amsfonts,amssymb,graphicx,amsthm}
\usepackage{fancyhdr}
\usepackage{geometry}
\geometry{margin=1in}
\pagestyle{fancy}
\fancyhf{}
\rhead{Multi-Causal Intelligence}
%\lhead{Knuthian Style}
\rfoot{\thepage}
\usepackage{titlesec}
\titleformat{\section}{\normalfont\Large\bfseries}{\S\thesection}{1em}{}

\title{\textbf{Multi-Causal Intelligence (MCI)}\\\large Foundations and Structures}
\author{Kireva [\emph{Ens Imaginalis}]}
\date{}

\newtheorem{definition}{Definition}
\newtheorem{theorem}{Theorem}

\begin{document}

\maketitle

\section{Introduction}

The theory of \textbf{Multi-Causal Intelligence (MCI)} arises from the
intersection of computer science, category theory, and causal inference. It
seeks to formalize and implement agents capable of recursively reasoning across
multiple layers of causal abstraction.

While conventional AI systems operate within singular or static causal
frameworks, MCI proposes a dynamic, compositional, and multi-layered approach,
where causal relationships are not fixed but generated, transformed, and
recursively reasoned over.

\section{Causal Graphs and Structural Foundations}
\label{sec:causal-graphs-structural-foundations}

Let us define a \textbf{causal graph} \( G = (V, E) \), where \( V \) is a set
of nodes (representing variables or events), and \( E \subset V \times V \) is
a set of directed edges (representing causal relationships).

A causal model \( \mathcal{M} \) is defined as:
\[
\mathcal{M} = \langle G, \mathcal{F}, \mathcal{P} \rangle
\]
where:
\begin{itemize}
    \item \( G \) is a causal graph.
    \item \( \mathcal{F} = \{ f_v \mid v \in V \} \) is a set of structural functions.
    \item \( \mathcal{P} \) is a probability distribution over exogenous variables.
\end{itemize}

MCI extends this notion by considering multiple causal models \( \mathcal{M}_1,
\dots, \mathcal{M}_n \), with a meta-structure defining relationships between
them.

\subsection{The Multiway Causal Structure}

We define a \textbf{multiway causal system} \( \mathcal{W} \) as:
\[
\mathcal{W} = ( \{ \mathcal{M}_i \}_{i \in I}, \mathcal{T} )
\]
where \( \mathcal{T} : \mathcal{M}_i \rightarrow \mathcal{M}_j \) represents transitions such as interventions or structural updates.

This defines a branching, non-deterministic causal space akin to Wolfram's multiway systems.

\subsection{Categorical Abstractions}

In MCI, causal graphs and transformations are modeled in a category \(
\mathbf{Causal} \). Each causal model is an object, and each morphism
represents a transformation preserving causal structure.

\textbf{Definition:} A \textit{Causal Category} \( \mathbf{Causal} \) has:
\begin{itemize}
    \item Objects: causal models \( \mathcal{M} \)
    \item Morphisms: functorial transformations
    \item Composition: models composite reasoning
\end{itemize}

Functors, 2-categories, limits, and colimits all play roles in MCI for aligning
and transforming knowledge layers.

\subsection{Recursive Reasoning Agents}

Define a recursive agent \( \mathcal{A} \) by:
\[
\mathcal{A} = \langle \Sigma, \delta, \mu, \rho \rangle \]
where:
\begin{itemize}
    \item \( \Sigma \): internal symbolic state
    \item \( \delta \): transition on observations
    \item \( \mu \): model proposal function
    \item \( \rho \): recursive introspection
\end{itemize}

Agents not only traverse models but propose and revise them through symbolic
recursion.

\subsection{Implementation Outline}

The implementation of MCI includes:
\begin{itemize}
    \item \textbf{Graph Layer}: DAGs with interventions
    \item \textbf{Multiway Engine}: branching causal paths
    \item \textbf{Category Engine}: transformations and abstractions
    \item \textbf{Agentic Kernel}: recursive and symbolic agents
\end{itemize}

This supports both symbolic and probabilistic reasoning and meta-model
operations.

\subsection{Applications and Future Work}

MCI supports:
\begin{itemize}
    \item Scientific hypothesis generation
    \item Multi-agent epistemic alignment
    \item Adaptive ontologies
    \item Reflexive symbolic-causal fields
\end{itemize}

Future work involves recursive agents that not only navigate causal spaces but
\textit{generate} and \textit{transcend} them.


\section{\textbf{Constructing a Minimal MCI Agent}\\\large Recursive Traversal of Multiway Causal Systems}


\subsection{Agent Architecture}

We define a minimal recursive agent \( \mathcal{A} = \langle \Sigma, \delta,
\mu, \rho \rangle \) as in
\S~\ref{sec:causal-graphs-structural-foundations}. The goal of this
section is to instantiate \( \mathcal{A} \) with computable structures and
demonstrate recursive traversal across a multiway causal system.

\begin{definition}

A \textbf{symbolic state space} \( \Sigma \) is a finite set of strings \( s
\in \Sigma \), each representing an encoded causal hypothesis or intervention
path.

\end{definition}

\begin{definition}

The \textbf{transition function} \( \delta: \Sigma \times \mathcal{M}
\rightarrow \Sigma \) is a computable function that updates the agent's
internal state based on observations from a causal model \( \mathcal{M} \).

\end{definition}

\begin{definition}

The \textbf{model proposal function} \( \mu: \Sigma \rightarrow \mathcal{M} \)
maps a symbolic state to a new causal model via decoding and instantiation.

\end{definition}

\begin{definition}

The \textbf{recursive operator} \( \rho: \Sigma \rightarrow \mathcal{P}(\Sigma)
\) generates a set of internal reflections, supporting higher-order
introspection and structural correction.

\end{definition}

\subsection{Minimal Construction}

Let us define a finite set of causal models \( \mathcal{M}_0, \mathcal{M}_1,
\dots, \mathcal{M}_n \), each represented as a DAG over Boolean variables. The
models evolve via a multiway system \( \mathcal{T} \) where each transition
corresponds to a possible intervention or structural mutation.

Let the symbolic state space \( \Sigma \) encode tuples \( (i, h) \), where:
\begin{itemize}
    \item \( i \) is the current model index in \( \{0, \dots, n\} \)
    \item \( h \) is a hash of past interventions
\end{itemize}

The agent iteratively updates \( (i, h) \rightarrow (j, h') \) via:
\begin{enumerate}
    \item Querying \( \mathcal{M}_i \) for inconsistencies
    \item Selecting a transition \( \mathcal{T}(i) = j \)
    \item Updating \( h \mapsto h' \) via an introspective function \( \rho \)
\end{enumerate}

\section{Recursive Traversal Algorithm}

Let us define the recursive traversal algorithm as:

\begin{center}
\fbox{\parbox{0.9\textwidth}{
\textbf{Algorithm: RecursiveTraverse}\\
\textbf{Input:} Initial state \( s_0 = (i_0, h_0) \), depth limit \( d \)\\
\textbf{Output:} Set of reachable states \( \Sigma' \)\\
\begin{enumerate}
    \item Initialize \( \Sigma' \leftarrow \{s_0\} \)
    \item For \( k = 1 \) to \( d \):
    \begin{itemize}
        \item For each \( s = (i, h) \in \Sigma' \):
        \begin{enumerate}
            \item Query \( \mathcal{M}_i \) and update state via \( \delta \)
            \item Apply \( \rho(s) \rightarrow \{s_1, \dots, s_m\} \)
            \item Apply \( \mu(s_j) \) to construct new models \( \mathcal{M}_{j} \)
            \item Extend \( \Sigma' \leftarrow \Sigma' \cup \{s_1, \dots, s_m\} \)
        \end{enumerate}
    \end{itemize}
    \item Return \( \Sigma' \)
\end{enumerate}
}}
\end{center}

\subsection{Soundness and Coherence}

\begin{theorem}

If \( \delta \), \( \mu \), and \( \rho \) are computable and \( \mathcal{T} \)
is finite, then RecursiveTraverse terminates and explores a sound subspace of
the multiway system.

\end{theorem}

\begin{proof}

Each recursive step operates on a finite symbolic state and terminates in
bounded time. Since \( \mathcal{T} \) and \( \Sigma \) are finite and traversal
depth is bounded by \( d \), the algorithm must halt. Soundness follows from
the well-defined update and introspection semantics.

\end{proof}

\begin{theorem}

If causal models are aligned via functorial morphisms, then traversal paths
form a coherent diagram in the category \( \mathbf{Causal} \).

\end{theorem}

\begin{proof}

Let each model transition \( f: \mathcal{M}_i \rightarrow \mathcal{M}_j \) be a
morphism in \( \mathbf{Causal} \). Then compositions \( f \circ g \) satisfy
associativity and identity properties by category axioms. The recursive diagram
thus commutes.

\end{proof}

\subsection{Conclusion}

We have constructed a minimal recursive MCI agent and proven its sound
traversal within a bounded multiway causal space. Further work may generalize
the state space, introduce higher categories, and relax determinism for
non-linear inference chains.


\end{document}
